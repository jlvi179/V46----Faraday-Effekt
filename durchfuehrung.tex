\section{Durchführung}
\label{sec:Durchführung}

\subsection{Justierung}
Zuerst muss der Versuch justiert werden. Dazu muss die Kondensorlinse so ausgerichtet werden, dass das Licht parallel ausgerichtet ist und 
den Spalt in dem Magneten möglichst mittig trifft. Der Polarisator wird dafür aus dem Aufbau entfernet. 
Es wird außerdem überprüft, ob die Lichtstrahlen die Photowiderstände treffen. dafür können die beiden Lichtschutzhauben entfernt werden und 
das Prisma hinter dem Magneten sowie die Photodioden justiert werden. \\
\\
Der Polarisator wird jetzt wieder eingebaut und so gedreht, dass sich das Licht zwischen den beiden Photowiderständen hin- und herschalten lässt.
Zuletzt wird am Selektivverstärker die Frequenz des Lichtzerhackers $f = 450\,\unit{\hertz}$ eingestellt. Es wird einer der Photowiderstände 
an den Selektivverstärker angeschlossen. Es sollte nun eine scharfe Kurve zu erkennen sein. Wenn dies nicht der Fall ist, muss die Frequenz des 
Selektivverstärkers minimal angepasst werden. 

\subsection{Messung}
Zuerst wird die Faraday-Rotation an einer hochreinen GaAs-Probe ($d=5.11 \,\unit{\milli\meter}$) und an zwei 
n-dotierten Proben mit $N=1.2\cdot 10^{18}\,\unit{\centi\meter}^{-3}$ bzw. $N=2.8 \cdot 10^{18}\,\unit{\centi\meter}^{-3}$ und 
$d=1.36\,\unit{\milli\meter}$ bzw. $d=1.296\,\unit{\milli\meter}$ gemessen. Dazu wird jeweils die Lichtintensität am Oszilloskop
minimiert und der Winkel notiert. Für jede Probe werden alle neun Interferenzfilter sowie beide Magnetfeldpolungen 
vermessen. \\
\\
Zuletzt werden die beiden Photowiderstände sowie die Halterung für den Interferenzfilter entfernt und durch eine Hallsonde ersetzt. Mit dieser 
wird das Magnetfeld $B(z)$ in der Nähe des Luftspalts parallel zur Ausbreitungsrichtung des Lichts ausgemessen. 
\section{Diskussion}
\label{sec:Diskussion}
Die Messung des magnetischen Feldes mittels einer Hallsonde wurde in \autoref{sec:AuswB} analysiert. Die magnetische Feldstärke zeigt den 
erwarteten Verlauf, dass die diese in der Mitte des Magneten am stärksten ist -- also dort, wo sich die Probe befindet -- und nach außen hin 
stark abnimmt. Außerdem konnte die maximale Feldstärke bestimmt werden. \\
\\
Wenn die experimentell bestimmten effektiven Massen durch die Ruhemasse von Elektronen ausgedrückt werden, ergeben sich 
\begin{align*}
    m^*_1 &= (310 \pm 27)\cdot m_0 \; , \\
    m^2_2 &= (197\pm 9)\cdot m_0 \; .
\end{align*}
Diese können nun mit dem Literaturwert $m^* = 0.067\cdot m0$ \cite{effM} verglichen werden. Dabei ergeben sich Abweichungen von
\begin{align*}
    \Delta m^*_1 &= ( \pm )\,\% \; , \\
    \Delta m^*_2 &= ( \pm )\,\% \; .
\end{align*}

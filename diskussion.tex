\section{Diskussion}
\label{sec:Diskussion}

Einen Literaturwert für die effektive Masse der Ladungselektronen in n-dotiertem GaAs ist in \cite{effM} durch 
\begin{equation*}
    m^*_\text{lit}= (0.067) m_0
\end{equation*}
gegeben. Zu sehen ist also, dass sich die oben ermittelten Werte in der gleichen Größenordnung befinden, wie der Literaturwert.
Die prozentuale Abweichungen ergeben sich zu
\begin{align*}
    \frac{m^*_1-m*_\text{lit}}{m*_\text{lit}}&=(161\pm 23) \%\; ,\\
    \frac{m^*_2-m*_\text{lit}}{m*_\text{lit}}&=(66\pm 8) \% \; .
\end{align*}
Diese Abweichungen lassen sich im Allgemeinen durch mögliche Verunreinigungen oder Rauschen begründen. Auf einigen Interferenzfiltern befanden sich neben Fingerabdrücken auch 
Beschädigungen. Die dadurch bedingte Streuung der Messwerte führt dazu, dass einige Messungen von der Auswertung ausgeschlossen wurden, da deren Abweichung im Verhältnis zu denen der anderen wesentlich größer
sind. Gerade bei der ersten Probe hat dieser mehrere Messwerte unbrauchbar gemacht.

Neben den aufgeführten Mängel des Aufbaus kann auch ein ungenaues Justieren nicht ausgeschlossen werden. Einige Feststellschrauben sind nicht in der Lage das jeweilige Bauteil fest zu justieren, der Strahlengang wurde nur mit dem Auge 
abgeschätzt und auch die Minima und die daraus resultierenden Winkel wurden manuell abgelesen.

Allgemein kann also zusammengefasst werde, dass eine Vielzahl von systematischen Fehlern die Messergebnisse beeinflusst und verfälscht haben. Zu einer Verbesserung der Werte müsste folglich 
die Apparatur verändert und eine Möglichkeit gefunden werden, die Messwertnahme zu verbessern.
\section{Auswertung}
\label{sec:Auswertung}

\subsection{Fehlerrechnung}
\label{sec:Fehlerrechnung}
Für die Fehlerrechnung werden folgende Formeln aus der Vorlesung verwendet.
für den Mittelwert gilt
\begin{equation}
    \overline{x}=\frac{1}{N}\sum_{i=1}^N x_i ß\; \;\text{mit der Anzahl N und den Messwerten x} 
    \label{eqn:Mittelwert}
\end{equation}
Der Fehler für den Mittelwert lässt sich gemäß
\begin{equation}
    \increment \overline{x}=\frac{1}{\sqrt{N}}\sqrt{\frac{1}{N-1}\sum_{i=1}^N(x_i-\overline{x})^2}
    \label{eqn:FehlerMittelwert}
\end{equation}
berechnen.
Wenn im weiteren Verlauf der Berechnung mit der fehlerhaften Größe gerechnet wird, kann der Fehler der folgenden Größe
mittels Gaußscher Fehlerfortpflanzung berechnet werden. Die Formel hierfür ist
\begin{equation}
    \increment f= \sqrt{\sum_{i=1}^N\left(\frac{\partial f}{\partial x_i}\right)^2\cdot(\increment x_i)^2}.
    \label{eqn:GaussMittelwert}
\end{equation}


\subsection{Messung des Magnetfeldes}
\label{sec:AuswB}
Die aufgezeichneten Messwerte der magnetischen Flussdichte sind in \autoref{tab:BFeld} aufgetragen.
Es wurde die Flussdichte $B$ in Abhängigkeit der Auslenkung $r$ notiert.
\begin{table}
    \centering
    \caption{Messwerte der magnetischen Feldstärke in Abhängigkeit vom Abstand.}
    \begin{tabular}{c c  }
        \toprule
        Auslenkung $r \mathbin{/}\unit{\milli \meter}$& $B\mathbin{/}\unit{\milli \tesla}$\\
        \midrule
        86	&102\\
        88&	219\\
        90&	325\\
        92&	375\\
        94&	400\\
        96&	413\\
        98&	418\\
        100&417\\
        102&410\\
        104&396\\
        106&365\\
        108&304\\
        110&204\\
        112&105\\
        \bottomrule
    \end{tabular}
    \label{tab:BFeld}
\end{table}

\begin{figure}[H]
    \centering
    \includegraphics[width=0.8\textwidth]{build/BFeld.pdf}
    \caption{Magnetische Flussdichte in Abhängigkeit der Auslenkung.}
    \label{fig:BFeld}
\end{figure}
Das Maximum der Messwerte entspricht hierbei der Mitte der Spule, also genau der Ort an welchem später die Halbleiter-Proben platziert werden.
Der eingezeichnete Maximalwert beträgt $418\, \unit{\milli \tesla}$.

\subsection{Faraday-Effekt}
In \autoref{tab:winkel} werden die gemessenen Winkel für die drei Proben in Abhängigkeit von der Wellenlänge dargestellt. Dabei handelt es sich bei der 
ersten Probe um n-dotiertes GaAs  mit $N = 1.2 \cdot 10^{18}\,\unit{\centi\meter}^3$ und $d= 1.36\,\unit{\milli\meter}$. Die zweite Probe ist 
n-dotiertes GaAs mit $N = 2.8 \cdot 10^{18}\,\unit{\centi\meter}^3$ und $d= 1.296\,\unit{\milli\meter}$. Bei der letzten Probe handelt es sich um 
hochreines GaAs mit einer Dicke von $d= 5.11\,\unit{\milli\meter}$. \\
\\
\begin{table}
    \centering
    \caption{Gemessene Winkel in Abhängigkeit von der Wellenlänge.}
    \begin{tabular}{c c c c c c c }
        \toprule
        & \multicolumn{2}{c}{n-GaAs (1)} & \multicolumn{2}{c}{n-GaAs (2)} & \multicolumn{2}{c}{GaAs} \\
		\cmidrule(lr){2-3}\cmidrule(lr){4-5}\cmidrule{6-7}
		{$\lambda \mathbin{/} \unit{\micro\meter}$} &
        {$\theta_1 \mathbin{/} \unit{\degree}$} & {$\theta_2 \mathbin{/} \unit{\degree}$} &
        {$\theta_1 \mathbin{/} \unit{\degree}$} & {$\theta_2 \mathbin{/} \unit{\degree}$} &
        {$\theta_1 \mathbin{/} \unit{\degree}$} & {$\theta_2 \mathbin{/} \unit{\degree}$} \\
		\midrule
        1.060& 79.90 & 88.75  & 78.05  &  88.68 & 71.17  &  93.80 \\
        1.290& 78.42 &  86.27 &  78.18 &  86.22 &  73.90 &  89.55 \\
        1.450& 81.22 &  87.73 &  83.11 &  88.02 &  78.42 &  88.97 \\
        1.720& 79.25 &  86.01 &  77.23 &  87.28 &  78.60 &  86.57 \\
        1.960& 74.67 &  81.15 &  73.72 &  84.57 &  73.22 &  80.83 \\
        2.156& 74.10 &  86.97 &  70.72 &  82.60 &  74.35 &  80.37 \\
        2.340& 46.95 &  53.53 &  44.88 &  56.82 &  49.03 &  53.00 \\
        2.510& 28.45 &  35.67 &  27.30 &  40.77 &  30.67 &  34.38 \\
        2.650& 67.27 &  74.90 &  64.18 &  82.80 &  68.18 &  72.38 \\
        \bottomrule
    \end{tabular}
    \label{tab:winkel}
\end{table}
Im folgenden soll der auf die Einheitslänge normierte Rotationswinkel in Abhängigkeit von der Wellenlänge graphisch aufgetragen werden. Dazu 
wird zuerst die Differenz der beiden Polungen gebildet und im Anschluss auf die zweifache Dicke der Probe normiert. 
Die Berechnung der Differenz der normierten Drehwinkel erfolgt mittels
\begin{equation*}
    \Delta \theta = \lvert \frac{1}{2d}(\theta_1 - \theta_2) \rvert \,\, .
\end{equation*}
Die graphische Darstellung für die einzelnen Proben findet sich in \autoref{fig:ndotiert1}, \autoref{fig:ndotiert2} und \autoref{fig:hochrein}.
\begin{figure}[H]
    \centering
    \includegraphics[width=0.8\textwidth]{build/ndotiert1.pdf}
    \caption{Differenz der normierten Drehwinkel für Probe 1.}
    \label{fig:ndotiert1}
\end{figure}

\begin{figure}[H]
    \centering
    \includegraphics[width=0.8\textwidth]{build/ndotiert2.pdf}
    \caption{Differenz der normierten Drehwinkel für Probe 2.}
    \label{fig:ndotiert2}
\end{figure}

\begin{figure}[H]
    \centering
    \includegraphics[width=0.8\textwidth]{build/hochrein.pdf}
    \caption{Differenz der normierten Drehwinkel für hochreines GaAs.}
    \label{fig:hochrein}
\end{figure}

\subsection{effektive Masse}